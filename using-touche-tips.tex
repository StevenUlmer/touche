\documentclass{article}

\usepackage{listings}
\usepackage{filemod}

\begin{document}
	\title{
		{\LARGE Programming Contest Tips} \\
		{\large Using Taylor University's Touche server}}
	\date{}
	\maketitle{}
	
	\section{Submitting}
	\subsection{How do I submit a problem?}
	\begin{enumerate}
		\item Click on ``Submissions"
		\item Choose the problem you want to respond to from the drop down in the center of the screen
		\item Choose your source file (.c, .cpp, .java, or .py)
		\item Click ``Submit Solution"
	\end{enumerate}
	
	\subsection{When do I get to know how I did?}
	After you submit your file, your submission should have a a status of ``Submitted for judging" or ``Pending."
	If this is the case, sit tight. We are working on judging your submission. 
	It should take 1-5 minutes.
	
	\section{Requesting a Clarification}
	In the interest of fairness, we ask that you do not ask the judges questions directly 
	but instead use the Clarifications page to ask clarification questions you may have. 
	
	\subsection{Ask judges these kind of questions}
	\begin{itemize}
		\item How do I submit my code?
		\item The wording on this question is ambiguous. Can you clarify?
		\item What does a ``Forbidden word" error mean?
	\end{itemize}

	\subsection{Ask your team these kinds of questions}
	\begin{itemize}
		\item What is the syntax for this?
		\item How do I solve this problem?
		\item What is causing this error?
	\end{itemize}
	
	\section{Coding in C/C++}
	\subsection{Sample program}
	\lstset{language=C}
	\lstset{breaklines=true}
	\begin{lstlisting}
#include <stdio.h>

//adds two numbers
int main(int argc, char **argv){
	int one, two;
	scanf("%d %d", &one, &two);
	printf("%d\n", one+two);
}
	\end{lstlisting}

	\subsection{Things that cannot appear anywhere in your code:}
	\begin{lstlisting}
system fstream open __asm__ socket connect accept listen mmap
	\end{lstlisting}

	\subsection{Headers}
	The contest software (touche) will remove all of your \#include statements 
	and replace them with the following. 
	This means that you cannot reference any headers that do not appear in these lists.

	\subsubsection{C}
	\begin{lstlisting}
stdio.h, stdlib.h, string.h, math.h, malloc.h, ctype.h, assert.h, limits.h
	\end{lstlisting}
	
	\subsubsection{C++}
	\begin{lstlisting}
cassert, cstdio, cstdlib, cstring, cmath, climits, iostream, sstream, iomanip, string, new, stdexcept, cctype, list, queue, stack, vector, map, iterator, bitset, algorithm, iomanip, set
	\end{lstlisting}
	
	\section{Coding in Java}
	\subsection{Sample program}
	\lstset{language=java}
	\begin{lstlisting}
import java.util.Scanner;

//cannot read "public class Main"
class Main{ 
	//adds two numbers
	public static void main(String[] args){
		Scanner reader = new Scanner(System.in); 
		int one, two;
		one = reader.nextInt();
		two = reader.nextInt();
		System.out.println(one + two);
	}
}
	\end{lstlisting}
	
	\subsection{Packages}
	The contest software (touche) makes these packages avalible for import. 
	This means that you cannot reference any packages that do not appear in this list.

	\begin{lstlisting}
java.lang.*
java.io.*
java.util.*
java.math.*
	\end{lstlisting}
	
	\section{Coding Python 2/3}
	\subsection{Sample program}
	\lstset{language=Python}
	\begin{lstlisting}
# Adds two numbers

line = input()
one, two = line.split()
one = int(one)
two = int(two)
print(one + two)
	\end{lstlisting}
	
	\subsection{Modules}
	You may import the following modules in your code:
	\begin{lstlisting}
numbers, math, cmath, decimal, fractions, random, itertools, functools, operator
	\end{lstlisting}

\end{document}
